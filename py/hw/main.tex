\documentclass{article}
\usepackage[utf8]{inputenc}

\title{CSE 15: Discrete Mathematics\\Homework 3}
\author{Nathan Vasquez}
\date{March 1, 2020}

\usepackage{enumitem}

\begin{document}

\maketitle

\section*{Knights and Knaves}

\begin{itemize}
    \item Scenario 1:\\There are two possibilities for $P$ and $Q$ in this scenario. Either $P$ and $Q$ are Knaves, or $P$ is a Knight and $Q$ is a Knave.
    \item Scenario 2:\\
    \begin{center}
    \begin{tabular}{|c|c|c|}
    \hline
    $B$ & $A$ & $B \to A$ \\
    \hline
    0 & 0 & 1 \\
    0 & 1 & 1 \\
    1 & 0 & 0 \\
    1 & 1 & 1 \\
    \hline
    \end{tabular}
    \end{center}\\
    $A$ Cannot be a Knight because then his statement would contradict him being a Knight, therefore $A$ has to be a knave, which means that his statement has to be false. The only scenario in which $A$'s statement is false is when $B$ is a Knight and $A$ is a Knave

\end{itemize}
\section*{Logical Identities}
\begin{enumerate}
    \item $\lnot(p \to (q \to p))$ \\
    $p \land \lnot (q \to p)$\ Implication Law \\
    $p \land q \land \lnot p$ Implication Law \\
    $p \land \lnot p \equiv F$\\
    $F \land q \equiv F$
    \item $\lnot ((p \land q) \to (p \lor q))$\\
    $((p \land q) \land  \lnot(q \lor p))$. Implication Law \\
    $(p \land q) \land (\lnot q \land \lnot p)$ Negation Law\\
    $(p \land \lnot p) \land (q \land \lnot q)$ Rearrange\\
    $(p \land \lnot p) \equiv F
    \\(q \land \lnot q) \equiv F
    \\ F \land F \equiv F$
\end{enumerate}

\section{Logical Equivalence}
\begin{enumerate}
    \item $p \to (q \to r) \equiv (p \land q) \to r$\\
    $\lnot p \lor (q \to r) \equiv \lnot(p \land q) \lor r\\ 
    \lnot p \lor \lnot q \lor r \equiv \lnot p \lor \lnot q \lor r\\$
    They are equivalent
    \item $p \to (q \to r) \ne (p \to q) \to r$\\
    $\lnot p \lor (q \to r) \ne \lnot(p \to q) \lor r$\\
    $\lnot p \lor \lnot q \lor r \ne p \land \lnot q \lor r$
\end{enumerate}
\section{Logical Consequence}
\begin{enumerate}
    \item Jimmy is smart\\Smart people are rich\\
    \line(1,0){100}\\Jimmy is rich\\ \\This is valid because smart people in this case is a subset of rich, and Jimmy is a subset of smart people therefore Jimmy is rich.
    \item Islands are surrounded by water\\Puerto Rico is surrounded by water\\
    \line(1,0){100}\\
    Puerto Rico is an island\\
    
    \\ This is Invalid because Puerto Rico because Puerto Rico and Islands have similar characteristics but Puerto Rico is not always a subset of Islands, therefore it is Invalid.
\end{enumerate}

\end{document}
